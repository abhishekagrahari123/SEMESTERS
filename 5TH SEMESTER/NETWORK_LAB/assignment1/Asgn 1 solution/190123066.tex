\documentclass[12pt]{article}
\usepackage{fancyhdr}
\usepackage{graphicx}
\graphicspath{{images/}}

\tolerance = 1
\emergencystretch=\maxdimen
\hyphenpenalty=10000
\hbadness=10000

\pagestyle{fancy}
\fancyhf{}
\lhead{CS342}
\rhead{Computer Networks Laboratory}
\rfoot{190123066}
\lfoot{Abhishek Agrahari}
\cfoot{\thepage}

\title{CS342 : Computer Networks Laboratory\\Assignment 1}
\date{}
\author{\LARGE\textbf{Abhishek Agrahari}\\\\\LARGE\textbf{190123066}\\\\Mathematics and Computing\\\\IIT Guwahati}
\newpage

\begin{document}
\newpage
\maketitle
\clearpage

\bgroup\obeylines
\textbf{Ans 1}-

\textbf{a)} -c count =$>$ specifies the number of echo requests, as indicated by the count variable, to be sent.

\textbf{b)} -i Wait =$>$ Waits the number of seconds specified by the Wait variable between the sending of each packet. The default is to wait for one second between each packet.

\textbf{c)} -l Preload =$>$ Defines the number of packets to send without waiting for a reply. To specify a value higher than 3, you need superuser permissions.

\textbf{d)} -s PacketSize =$>$ Specifies the number of data bytes to be sent.If the payload size is set to 32 bytes, total packet size becomes 40 bytes when combined with the 8 bytes of ICMP header data.\\

\textbf{Ans 2}-
\textbf{IP 1 -} www.google.com US 
\textbf{IP 2 -} www.codeforces.com sweden
\textbf{IP 3 -} www.huawei.com china
\textbf{IP 4 -} www.india.gov.in india
\textbf{IP 5 -} www.leetcode.com  
\textbf{IP 6 -} www.whatsapp.com germany
\egroup
\begin{center}
\begin{tabular}{ |c|c|c| } 
 \hline
 cell1 & cell2 & cell3 \\ 
 cell4 & cell5 & cell6 \\ 
 cell7 & cell8 & cell9 \\ 
 \hline
\end{tabular}
\end{center}





\end{document}