\documentclass[12pt]{article}
\usepackage[utf8]{inputenc}
\usepackage{fancyhdr}
\usepackage{fancyvrb}
\usepackage{textcomp}
\usepackage{subfigure}
\usepackage{float}
\usepackage{graphicx}
\graphicspath{{images/}}
\usepackage{enumitem}
\usepackage{amsmath}
\usepackage{geometry}
\geometry{
	top  = 17mm,
	left = 10 mm ,
	right = 10 mm ,
	bottom = 17mm
}
\pagestyle{fancy}
\lhead{MA-374}
\rhead{Financial Engineering Lab}
\makeatother
\renewcommand{\labelitemii}{*}
\renewcommand{\labelitemiii}{$\circ$}
\renewcommand{\labelitemiv}{$\bullet$}

\title{Assignment 7}
\author{Abhishek Agrahari\\190123066}
\date{}
\begin{document}
\maketitle

\section*{Question 1} 
The solution to the Black-Scholes-Merton equation with terminal condition and boundary conditions is - 
$$ c(t,x) = x.N(d_+(T-t,x)) - K.e^{-r(T-t)}.N(d_-(T-t,x)) \text{ where 0} \leq t < \text{T and x} > 0.$$
where $$ d_{\pm}(\tau,x) = \frac{1}{\sigma \sqrt{\tau}}\left[ \log\frac{x}{K} + (r \pm \frac{\sigma^2}{2}) \right]$$
and $N$ is the  cumulative standard normal distribution
$$N(y) = \frac{1}{\sqrt{2\pi}}\int_{-\infty}^y e^{\frac{-z^2}{2}} dz = \frac{1}{\sqrt{2\pi}}\int_{-y}^\infty e^{\frac{-z^2}{2}} dz$$

\noindent By put call parity we know that - 
$$ f(t,x) = x - e^{-r(T-t)}K = c(t,x) - p(t,x)$$
where $f$ is the value of a forward quantity at time $t \in [0,T]$ where stock price at time $t$ is $S(t) = x$.
Therefore, 
\begin{align*}
p(t,x) &= x.(N(d_+(T-t,x))-1) - K.e^{-r(T-t)}.(N(d_-(T-t,x))-1)\\
 &= K.e^{-r(T-t)}.N(-d_-(T-t,x)) - x.N(-d_+(T-t,x))
\end{align*}

\noindent Using above formulas for $c(t,x)$ and $p(t,x)$, prices for European call and put option is computed for \\ 0 $\leq t <$ T. For $t = T$, $c(t,x) = max(x-K, 0)$ and $p(t,x) = max(K-x,0)$
\section*{Question 2}
\begin{itemize}
\item 2d plots for the Value of options $(c(t,x))$ is plotted by varying price of the underlying at a given time. This plot is plotted for $t =$ [ 0.0, 0.2, 0.4, 0.6, 0.8, 1.0 ]
\begin{itemize}
	\item By increasing the price of the underlying, call prices increases while put prices decreases.
	\item As time to maturity $(T-t)$ decreases value of the options decreases.
\end{itemize}
\item 3d scatter plots are plotted by varying time and stock price at that time simultaneously.
\begin{itemize}
\item Plots are suitably rotated to get a better view.
\end{itemize}

\end{itemize}
\begin{figure}[H]
     \begin{center}
%
        \subfigure {%
            \includegraphics[width=4in, height = 3in]{q2/fig1.png}
        }%
        \subfigure {%
           \includegraphics[width=4in, height = 3in]{q2/fig2.png}
        }\\ %  ------- End of the first row ----------------------%
        \subfigure {%
            \includegraphics[width=4.1in, height = 3.5in]{q2/fig3.png}
        }%
        \subfigure{%
            \includegraphics[width=4in, height = 3.5in]{q2/fig4.png}
        }%
    \end{center}
\end{figure}
\vspace{-60pt}
\section*{Question 3}
\begin{itemize}
	\item Surface plots are plotted by varying time and stock price at that time simultaneously.
	\item These plots are the surface plot version of the 3d scatter plots obtained in question 2.
	\item Plots are suitably rotated to get a better view.
\end{itemize}

\begin{figure}[H]
     \begin{center}
        \subfigure {%
            \includegraphics[width=4in, height = 3.3in]{q3/fig1.png}
        }%
        \subfigure {%
           \includegraphics[width=4in, height = 3.3in]{q3/fig2.png}
        }
    \end{center}
\end{figure}
\vspace{-50pt}
\section*{Question 4}
\begin{itemize}

\item I have done the senstivity analysis for the value of European call and put option as a function of following model parameters
	\begin{itemize}
		\item Strike Price ($K$)
		\item Risk Free Return ($r$)
		\item Sigma ($\sigma$)
		\item Time of expiration ($T$)
	\end{itemize}

\item 2d plots are plotted assuming price of the underlying asset as 1 at different time points. All other values are taken as given in question 2, i.e. $K$ = 1, $T$ = 1, $r$ = 0.05, $\sigma$ = 0.6.

\item One by one all these model parameters are varied and options price at given point of time is plotted. 
\end{itemize}

\begin{figure}[H]
     \begin{center}
%
        \subfigure {%
            \includegraphics[width=4in, height = 3in]{q4/fig1.png}
        }%
        \subfigure {%
           \includegraphics[width=4in, height = 3in]{q4/fig2.png}
        }\\ %  ------- End of the first row ----------------------%
    \end{center}
\end{figure}

\begin{figure}[H]
     \begin{center}
        \subfigure {%
            \includegraphics[width=4in, height = 3in]{q4/fig3.png}
        }%
        \subfigure {%
           \includegraphics[width=4in, height = 3in]{q4/fig4.png}
        }\\ %  ------- End of the first row ----------------------%
        \subfigure {%
            \includegraphics[width=4in, height = 3in]{q4/fig5.png}
        }%
        \subfigure {%
           \includegraphics[width=4in, height = 3in]{q4/fig6.png}
        }\\ %  ------- End of the first row ----------------------%
        \subfigure {%
            \includegraphics[width=4in, height = 3in]{q4/fig7.png}
        }%
        \subfigure {%
           \includegraphics[width=4in, height = 3in]{q4/fig8.png}
        }\\ %  ------- End of the first row ----------------------%
    \end{center}
\end{figure}
\begin{itemize}
\item Options price at t = 0 is plotted by varying two model parameters at a time.
\item Price of the underlying at t = 0, is assumed to be 1.
\end{itemize}
\vspace{-20pt}
\begin{figure}[H]
     \begin{center}
        \subfigure {%
            \includegraphics[width=4in, height = 2.85in]{q4/fig9.png}
        }%
        \subfigure {%
           \includegraphics[width=4in, height = 2.85in]{q4/fig10.png}
        }\\ %  ------- End of the first row ----------------------%
        \subfigure {%
            \includegraphics[width=4in, height = 2.85in]{q4/fig11.png}
        }%
        \subfigure {%
           \includegraphics[width=4in, height = 2.85in]{q4/fig12.png}
        }\\ %  ------- End of the first row ----------------------%
                \subfigure {%
            \includegraphics[width=4in, height = 2.85in]{q4/fig13.png}
        }%
        \subfigure {%
           \includegraphics[width=4in, height = 2.85in]{q4/fig14.png}
        }\\ %  ------- End of the first row ----------------------%
    \end{center}
\end{figure}
\begin{figure}[H]
     \begin{center}

        \subfigure {%
            \includegraphics[width=4in, height = 3in]{q4/fig15.png}
        }%
        \subfigure {%
           \includegraphics[width=4in, height = 3in]{q4/fig16.png}
        }\\ %  ------- End of the first row ----------------------%
        \subfigure {%
            \includegraphics[width=4in, height = 3in]{q4/fig17.png}
        }%
        \subfigure {%
           \includegraphics[width=4in, height = 3in]{q4/fig18.png}
        }\\ %  ------- End of the first row ----------------------%
        \subfigure {%
            \includegraphics[width=4in, height = 3in]{q4/fig19.png}
        }%
        \subfigure {%
           \includegraphics[width=4in, height = 3in]{q4/fig20.png}
        }\\ %  ------- End of the first row ----------------------%
    \end{center}
\end{figure}
\end{document}
