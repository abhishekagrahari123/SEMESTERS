\documentclass[12pt]{article}
\usepackage[utf8]{inputenc}
\usepackage{fancyhdr}
\usepackage{fancyvrb}
\usepackage{textcomp}
\usepackage{float}
\usepackage{graphicx}
\graphicspath{{images/}}
\usepackage{enumitem}
\usepackage{amsmath}
\usepackage{geometry}
\geometry{
	top  = 20mm,
	left = 10 mm ,
	right = 10 mm ,
	bottom = 20mm
}
\pagestyle{fancy}
\lhead{MA-322}
\rhead{Scientific Computing Lab}
\makeatother
\renewcommand{\labelitemii}{*}
\renewcommand{\labelitemiii}{$\circ$}
\renewcommand{\labelitemiv}{$\bullet$}

\title{Assignment 5}
\author{Abhishek Agrahari\\190123066}
\date{}
\begin{document}
\maketitle

\section{} 
\subsection{Code}
\begin{verbatim}
% we have to calculate g(1) which means integral 0 to 1 e^(-x^2)
clc;
clear;
a = 0;
b = 1;
actual_area = 0.74682413;
err = @(a,b) abs(a-b);

fprintf(" N \t\t\t\t Rrule \t\t\t\t Trule \t\t\t\t Srule " + ...
    "\t\t\t\t Er \t\t\t\t Et \t\t\t\t Es\n")
for N = [50, 100, 200]
    h = (b-a)/N;
    Rectangle_tot = 0;
    Trapezoidal_tot = 0;
    Simpson_tot = 0;
    for i = 1:N
        a1 = a + (i-1)*h;
        a2 = a + i*h;
        Rectangle_tot = Rectangle_tot + left_rectangle_rule(a1, a2);
        Trapezoidal_tot = Trapezoidal_tot + trapezoidal_rule(a1, a2);
        Simpson_tot = Simpson_tot + simpson_rule(a1,a2);
    end
    fprintf("%3.0f \t\t %.10f \t\t %.10f \t\t %.10f \t\t" + ...
        " %.10f \t\t %.10f \t\t %.10f\n", ...
        N, Rectangle_tot, Trapezoidal_tot, Simpson_tot, ...
        err(actual_area, Rectangle_tot), ...
        err(actual_area, Trapezoidal_tot), ...
        err(actual_area, Simpson_tot));
end


function val = f(x)
    val = exp(-(x^2));
end

function val = left_rectangle_rule(a,b)
    val = (b-a)*f(a);
end

function val = trapezoidal_rule(a,b)
    val = ((b-a)*(f(a) + f(b)))/2;
end

function val = simpson_rule(a,b)
    val = ((b-a)*(f(a) + 4*f((a + b)/2)+ f(b)))/6;
end
\end{verbatim}

\subsection{Output}
\begin{figure}[H]
\centering
\includegraphics[scale=0.75]{q1.png}
\end{figure}

\subsection{Note}
\begin{itemize}
\item I have used Left rectangle rule.
\end{itemize}

\subsection{Observations}
\begin{itemize}
  \item Error decreases as we go from Rectangle to Trapezoidal and Trapezoidal to Simpson rule.

  \begin{itemize}
  \item This observation is in accordance with the order of their errors as given below - 
  	\begin{itemize}
  		\item Rectangle Rule : $O(h)$
  		\item Trapezoidal Rule : $O(h^2)$
  		\item Simpson Rule : $O(h^4)$
  	\end{itemize}
  \end{itemize}

  \item Error decreases with increase in N.
  	\begin{itemize}
  		\item This is because as N increases, h decreases which is proportional to the error.
  	\end{itemize}
  \item Actual value of the integral has 8 significant digits and value computed through Simpson rule is correct to 8 digits. Therefore Simpson rule produced the correct integral upto the desired decimal places.
\end{itemize}

\clearpage
\section{}
\subsection{Code}
\begin{verbatim}
clc;
clear;
a = 0;
b = 1;
actual_area = 0.0808308690;
err = @(a,b) abs(a-b);

fprintf(" N \t\t\t    Trapezoidal \t   Corrected Trapezoidal \t  Err_Trapezoidal ...
 \t Err_Corrected_Trapezoidal\n")
for N = [1, 50, 100, 200]
    h = (b-a)/N;
    Trapezoidal_tot = 0;
    Corr_Trapezoidal_tot = 0;
    for i = 1:N
        a1 = a + (i-1)*h;
        a2 = a + i*h;
        Trapezoidal_tot = Trapezoidal_tot + trapezoidal_rule(a1, a2);
        Corr_Trapezoidal_tot = Corr_Trapezoidal_tot + corrected_trapezoidal_rule(a1,a2);
    end
    fprintf("%3.0f \t\t %.15f \t\t %.15f \t\t %.15f \t\t %.15f\n", ...
        N, Trapezoidal_tot, Corr_Trapezoidal_tot, ...
        err(actual_area, Trapezoidal_tot), err(actual_area, Corr_Trapezoidal_tot));
end

function val = f(x)
    val = (x^2)*exp(-2*x);
end

function val = f_prime(x)
    val = 2*exp(-2*x)*(x-x^2);
end

function val = trapezoidal_rule(a,b)
    val = ((b-a)*(f(a) + f(b)))/2;
end

function val = corrected_trapezoidal_rule(a,b)
    val = trapezoidal_rule(a,b) + ((b-a)^2)*(f_prime(a)-f_prime(b))/12;
end
\end{verbatim}

\subsection{Output}
\begin{figure}[H]
\centering
\includegraphics[scale=0.8]{q2.png}
\end{figure}

\subsection{Note}
\begin{itemize}
	\item N = 1 corresponds to non-composite method.
	\item In the observations section below I have used the following conventions - 
	\begin{itemize}
		\item $f(x) = x^2.e^{-2x}$
		\item $a = 0$
		\item $b = 1$
	\end{itemize}
\end{itemize}
\subsection{Observations}
\begin{itemize}
	  \item Error decreases with increase in N.
  	\begin{itemize}
  		\item This is because as N increases, h decreases which is proportional to the error.
  	\end{itemize}
	\item For all values of N, the value of integral computed through Trapezoidal and Corrected Trapezoidal rule is same.
	\begin{itemize}
		\item For N = 1, this is because error term given by the following expression, is zero.
			 $$ \text {\textit{Correction Term}} =\frac{(b-a)^2 (f'(a)-f'(b))}{12} $$
		\begin{itemize}
		 \item Derivative of f(x) is $f'(x) = 2.e^{-2x} (x - x^2)$ and $f'(0) = f'(1)$
		\end{itemize}
		\item For N $>$ 1, error term given by the following expression, is zero.		 			$$ \text {\textit{Correction Term}} =\frac{h^2 (f'(a)-f'(b))}{12} $$
		\begin{itemize}
		\item This highlights the fact that even in the composite case, correction term involves derivative at ends only. Derivative terms for the points in the middle cancels out. 
		\end{itemize}
		
	\end{itemize}
\end{itemize}

\clearpage
\section{}
\subsection{Code}
\begin{verbatim}
clc;
clear;
fun = @(x) exp((x.^2)./(-2));
err = @(a,b) abs(a-b);
for m = [1,2]
    fprintf("\nFor m = %d\n", m)
    fprintf(" N \t\t\t\t Srule \t\t\t\t\t Error \t\t\t\t\t Relative Error\n")
    a = -m;
    b = m;
    actual_prob = integral(fun, a, b)*(1/sqrt(2*pi));
    for N = [1, 50, 100, 200]
        h = (b-a)/N;
        Simpson_tot = 0;
        for i = 1:N
            a1 = a + (i-1)*h;
            a2 = a + i*h;
            Simpson_tot = Simpson_tot + simpson_rule(a1,a2);
        end
        Req_prob = Simpson_tot*(1/sqrt(2*pi));
        fprintf("%3.0f \t\t %.15f \t\t %.15f \t\t %.15f\n", N, Req_prob, ...
         err(actual_prob, Req_prob), err(actual_prob, Req_prob)/actual_prob);
    end

    fprintf("Actual Value of Required Probability : %.15f \n", actual_prob);
end

function val = f(x)
    val = exp((x^2)/(-2));
end

function val = simpson_rule(a,b)
    val = ((b-a)*(f(a) + 4*f((a + b)/2)+ f(b)))/6;
end
\end{verbatim}

\subsection{Output}
\begin{figure}[H]
\centering
\includegraphics[scale=0.9]{q3.png}
\end{figure}

\subsection{Note}
\begin{itemize}
	\item N = 1 corresponds to non-composite method.
	\item Relative error is calculated from the following formula -
	$$ Relative Error = \left| \frac{\text{\textit{Actual Integral - Computed Integral} } }{\text{\textit{Actual Integral}}}\right| $$ 
	\item Acutal Integral is calculated by an inbuilt function of matlab \begin{verbatim} integral(function_to_be_integrated, lower_limit, upper_limit)
	\end{verbatim}
\item Value of the probability is computed using the following formula - 
$$ P(-m\sigma \leq x \leq m\sigma) = \frac{1}{\sqrt{2\pi}}\int_{-m}^m e^{\frac{-z^2}{2}} dz$$ 
\end{itemize}
\subsection{Observations}
\begin{itemize}
\item Error decreases with increase in N.
  	\begin{itemize}
  		\item This is because as N increases, h decreases which is proportional to the error.
  	\end{itemize}
\item Relative error for m = 2 is greater than the relative error for m = 1. 
	\begin{itemize}
		\item Therefore we can say that as the length of integration increases, value of error also increases.
	\end{itemize}
\item Error for non composite method (N = 1) is quite high.
\end{itemize}
\end{document}
